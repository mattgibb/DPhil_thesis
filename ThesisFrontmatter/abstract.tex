{
\Large
\noindent\makebox[3in][l]{\xauthor}\hfill\makebox[3in][r]{Doctor of
Philosophy} \vskip 1pt
\makebox[3in][l]{\xcollege}\hfill\makebox[3in][r]{\xterm}
}

\vskip 1cm

{
\LARGE \bf
\begin{center}
{\xtitle}
\end{center}
}

{
\large\bf
\begin{center}
Abstract
\end{center}
}

\setlength{\baselineskip}{16truept}
TRANSFER THESIS
Computational modelling and simulation, in close interaction with experiments, has provided invaluable insight into the biochemical, mechanical and electrophysiological function and dysfunction of the heart. However, limitations in imaging techniques and computing resources have precluded the analysis of tissue architecture near the cellular scale and the effect of this architecture on cardiac function.

It is the wider aim of this thesis to develop a framework to investigate the role of microstructure in cardiac propagation dynamics and arrhythmogenesis. An initial modelling study elucidates the effect of blood vessels in sustaining arrhythmic episodes, and how the accurate modelling of fibre direction in the vicinity of the vessels mitigates this detrimental mechanism. A mathematical model of fibre orientation in a simple geometry around blood vessels has been developed, based on information obtained from highly detailed histological and MRI datasets. A simulation regime was chosen, guided by the vasculature extracted from whole heart MRI images, to analyse ventricular wavefront propagation for different orientations and positions of blood vessels. Our results demonstrate not only that the presence of the blood vessels encourages curvature in the activation wavefront around the blood vessels, but further that vessels act to restrict and prolong phase singularities. When compared to a more simplistic implementation of fibre orientation, the model is shown to weaken wavefront curvature and reduce phase singularity anchoring. Having established the importance of microstructural detail in computational models, it seems expedient to generate accurate data in this regard. The first steps have been taken merging MRI and histological images, in order to present the first 3-D sub-cellular resolution images of cardiac tissue, segmented by tissue type. Models including this detail will be developed and simulation will yield a deeper understanding of the role of microstructure in arrhythmia.
TRANSFER THESIS


CONFIRMATION REPORT
    Sudden cardiac death is the leading global cause of mortality, accounting for 39\% of all deaths in the UK. The economic cost due to the disease totals \pounds9 billion in this country alone. Ventricular fibrillation is a central aspect of many of these fatalities. However, many of the fundamental mechanisms underlying their onset, maintenance and termination are poorly understood, and effective therapies based on this understanding remain out of reach.

  Over the last century, experimental investigations into cardiac function during physiological and pathological conditions have allowed the general processes which produce arrhythmias to be characterised. In recent decades, experimental investigations have been complemented by computational and mathematical models which aim to simulate the biochemical, mechanical and electrophysiological function and dysfunction of the heart. The frequent validation of computational models with experimental data and use of modelling to guide new experimentation is bringing a new level to the understanding of cardiac function. It is now possible to assess the importance of small scale anatomical structure on large-scale cardiac behaviour. However, limitations in imaging techniques and computing resources have precluded the analysis of tissue architecture near the cellular scale and the effect of this architecture on cardiac behaviour.

    It is the aim of this thesis to characterise cardiac microstructure and to investigate its functional importance in cardiac propagation dynamics and arrhythmia. An initial modelling study in an idealised geometry will motivate the inclusion of microstructural detail into models for simulation. Image processing tools and techniques are developed to integrate images from several modalities and characterise this detail. Electrophysiological simulations of models derived from the images will provide insight into the functional consequences of microstructure. Scrutiny will be focused around several key strongly heterogeneous areas: epicardial arteries, papillary insertions, the apex and the junction of the septum with the myocardium.
CONFIRMATION REPORT

