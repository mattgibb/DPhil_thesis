%!TEX root = ../thesis.tex
{
\Large
\noindent\makebox[3in][l]{\xauthor}\hfill\makebox[3in][r]{Doctor of
Philosophy} \vskip 1pt
\makebox[3in][l]{\xcollege}\hfill\makebox[3in][r]{\xterm}
}

\vskip 1cm

{
\LARGE \bf
\begin{center}
{\xtitle}
\end{center}
}

{
\large\bf
\begin{center}
Abstract
\end{center}
}

\setlength{\baselineskip}{16truept}
Computational modelling and simulation, in close interaction with experiments, has provided invaluable insight into the biochemical, mechanical and electrophysiological function and dysfunction of the heart. However, limitations in imaging techniques and computing resources have precluded the analysis of tissue architecture near the cellular scale and the effect of this architecture on cardiac function.

It is the wider aim of this thesis to develop a framework to characterise cardiac microstructure and to investigate the role of microstructure in cardiac propagation dynamics and arrhythmogenesis. An initial modelling study elucidates the effect of blood vessels in sustaining arrhythmic episodes, and how the accurate modelling of fibre direction in the vicinity of the vessels mitigates this detrimental mechanism. A mathematical model of fibre orientation in a simple geometry around blood vessels has been developed, based on information obtained from highly detailed histological and MRI datasets. A simulation regime was chosen, guided by the vasculature extracted from whole heart MRI images, to analyse ventricular wavefront propagation for different orientations and positions of blood vessels. Our results demonstrate not only that the presence of the blood vessels encourages curvature in the activation wavefront around the blood vessels, but further that vessels act to restrict and prolong phase singularities. When compared to a more simplistic implementation of fibre orientation, the model is shown to weaken wavefront curvature and reduce phase singularity anchoring. Having established the importance of microstructural detail in computational models, it seems expedient to generate accurate data in this regard. An automated registration toolchain is developed to reconstruct histological slices based on coherent block face volumes, in order to present the first 3-D sub-cellular resolution images of cardiac tissue. Although mesoscopic geometry is faithfully reproduced throughout much of the dataset, low levels of transformational noise obfuscate tissue microstructure. These distortions are all but eradicated by a novel transformational diffusion algorithm, with characteristics that outperform any previous method in the literature in this domain, with respect to robustness, conservation of geometry and extent of information transfer. Progress is made towards extracting microstructural models from the resultant histological volumes, with a view to incorporating this detail into simulations and yielding a deeper understanding of the role of microstructure in arrhythmia.
