%!TEX root = ../thesis.tex

This chapter contributes a quantitative demonstration that the definition of microstructure affects electrical simulation results. We generate both a simplistic and a histologically based model of fibre direction around blood vessels of the proportion of epicardial coronary arteries, and use the models to construct idealised cuboid sections of ventricular wall containing transmural and epicardial vessels. We conduct electrophysiological simulations and compare the simplistic and the histologically based models, in order to distinguish the effects of the vessel cavity with those of the surrounding fibres. We also examine the effect of bidomain vs. monodomain simulations. We show that the activation patterns around large blood vessels are similar for bidomain and monodomain simulations. We conclude that inhomogeneities in cardiac tissue such as blood vessels can cause sharp wavefront curvature. We find that this curvature is less pronounced when fibre direction is modelled accurately around the vessels, as the wavefront is guided around the vessel by the curving fibres. Finally, we find that contrary to what we had hypothesised, negotiation of fibres around vessels actually diminishes vessel anchoring by funnelling the wavefront around the vessel, lessening its retarding and fragmenting effect. This work was presented at a plenary talk at Functional Imaging and Modelling of the Heart 2009 \cite{Gibb2009}.
