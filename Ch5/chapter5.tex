
\section{Introduction}
\label{sec:review:introduction}
CONFIRMATION REPORT
The aim in this chapter is to develop a computational representation of a blood vessel and the tissue surrounding it, in order to elucidate wavefront behaviour below the epicardium and the full 3-D interaction of excitation waves with intramural vessels. In this way, we attempt to build on the 2-D experimental findings from optical and electrical mapping studies, whilst complementing their weaknesses. We aim to clarify whether the inclusion of microstructure is important in whole ventricular models. We generate both a simplistic and a histologically based model of fibre direction around blood vessels, and use the models to construct idealised cuboid sections of ventricular wall containing transmural and epicardial vessels. We conduct electrophysiological simulations and compare the histologically based and the simplistic models, in order to distinguish the effects of the vessel cavity with those of the surrounding fibres. We also examine the effect of bidomain vs. monodomain simulations.

      We show that the activation patterns around blood vessels are similar for bidomain and monodomain simulations. We conclude that inhomogeneities in cardiac tissue such as blood vessels can cause sharp wavefront curvature. We find that this curvature is less pronounced when fibre direction is modelled accurately around the vessels, as the wavefront is guided around the vessel by the curving fibres. Finally, we find that contrary to what we had hypothesised, negotiation of fibres around vessels actually diminishes vessel anchoring by funnelling the wavefront around the vessel, lessening its retarding and fragmenting effect. These findings motivate the use of microstructure in whole ventricular models, and the challenge is set to characterise this microstructure and evaluate its effects with simulation.
CONFIRMATION REPORT



