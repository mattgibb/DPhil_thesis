%!TEX root = ../thesis.tex
\chapter{Introduction and Aims}
\dblspace

\section{Motivation}
\label{sec:intro:motivation}
  Sudden cardiac death is the leading cause of mortality in the world and in the UK, accounting for 39\% of all UK deaths. In 2008, over 191,000 died from heart and circulatory disease in the UK including 88,000 deaths
from coronary heart disease \cite{bhf2010}. Cardiovascular disease (CVD) caused more than 50,000 premature deaths, and more than one in four deaths in men before the age of 75 and one in five deaths in women before the age of 75 is from CVD. Just over \pounds15bn is spent each year on CVD, and it is estimated to cost the UK economy an annual \pounds30bn. An international research effort has emerged to develop a better understanding of cardiac function, and new insights are already guiding us toward better preventative measures and more effective treatments.

  Ventricular arrhythmia and fibrillation are central aspects to much of the fatality and debilitation caused by heart disease. Arrythmia, stemming from the Greek \emph{a-} and \emph{rhythm\'os} meaning `without flow', refers to a sustained and pathological departure from the natural contraction pattern of the heart, reducing its mechanical output. Fibrillation is a more extreme chaotic departure from organised contraction, often leading from arrhythmia, where output is all but eliminated, resulting in death within minutes if left untreated.
  
  The mechanical and electrophysiological operation of the heart is an extremely complex and highly controlled process, with many exquisite anatomical features orchestrating macroscopic behaviour in complex and non-linear ways. This tissue structure -- and in particular, microstructure -- is often peculiar to an individual, with small changes having large effects. For these reasons, many of the fundamental mechanisms underlying the onset, maintenance and termination of arrhythmia and fibrillation are poorly understood, and effective therapies based on this understanding remain out of reach.
  
  Over the last century, experimental investigations into cardiac function during physiological and pathological conditions have allowed the general processes which produce arrhythmias to be characterised. In recent decades, experimental investigations have been complemented by computational and mathematical models that aim to simulate the biochemical, mechanical and electrophysiological function and dysfunction of the heart \cite{StreeterJr1969,Hooks2002,Carusi2012}. The frequent validation of computational models with experimental data and use of modelling to guide new experimentation is bringing a new level to the understanding of cardiac function. This marriage of in silico and experimental investigation has made increasingly clear the important large-scale functional effects of small-scale anatomical structure.
  
  Very recently, several experimental techniques have provided high resolution data on both myocardial tissue structure \cite{Burton2006,Rutherford2012} and on healthy and pathological wavefront dynamics \cite{Valderrabano2003}. Histology is the process of serially sectioning ex vivo tissues and organs into extremely thin slices, and photographing each slice under microscope. It remains the gold standard for resolving tissue structure in full colour at a cellular scale. Despite providing two-dimensional images of unparalleled quality, each image is unrelated geometrically to any other in the series. Reconstructing coherent, smooth, three-dimensional volumes that are faithful to the overall shape of the sample before sectioning is the focus of a substantial body of literature, and each current method has its benefits and tradeoffs. 
  
  These advances notwithstanding, fine anatomical detail remains absent from the latest models and the effect of structure on wavefront dynamics has yet to be computationally explored. If microstructural effects on macroscopic function are to be investigated meaningfully, new methods must be developed to build accurate volumes from the data available to us upon which to base models for simulation. This investigation will bring us closer to a profound and predictive understanding of arrhythmia and will eventually lead to more effective therapies to combat acute heart disease.  
  
\section{Aims}
\label{sec:intro:aims}
  It is the wider aim of the thesis to develop a framework via which the role of microstructure in arrhythmia and electrophysiological pathology may be investigated, based upon information from the latest imaging techniques and leveraging state of the art high performance computing facilities. This chapter concludes by delineating the specific thesis aims and summarising the document structure. Guided by experimental findings, we aim to demonstrate the importance of blood vessels and myocardial anisotropy in electropropagation. We aim to construct accurate and detailed 3D images from high quality histology data at the tissue and whole-organ scale, and then to generate 3D tissue models based on this characterisation. To do so, new tools and algorithms need to be developed. Using these models, the way will be paved to quantify and classify the relative wavefront propagative changes due to epicardial vessels, myocardial fibres and tissue distributions. Applying both focussed and organ scale simulation, we may then explore how anatomical detail feeds into the prevention, the onset, the stabilisation and the subsequent rectification of arrythmias.

\section{Exegesis}
\label{sec:intro:exegesis}
  The next chapter introduces some central underpinnings of the thesis, with an overview of cardiac anatomy and electrophysiology, and of modelling and imaging techniques. A thorough literature review follows in Chapter~3, showcasing the experimental and computational evidence that inhomogeneities in tissue direct electrical dynamics. It also outlines the state of the art in computational modelling and simulation, and reviews the attempts made so far in the imaging of cardiac tissue.
  
  Chapter~4 describes an initial study that reflects experimental findings, showing that epicardial arteries anchor phase singularities and prolong arrhythmic episodes, but that this anchoring is reduced by the smooth negotiation of myocardial fibres around the vessels.
  
  Chapter~5 presents the automated construction of a geometrically precise 3D histological tissue map, based on a set of coherent block face reference images. An incremental relaxation of slice transform constraint, coupled with carefully tuned registration algorithms, leads to a robustly precise volume of a rat heart. An initialisation based on principal component analysis is shown to work very well in some circumstances, but deemed inappropriate for this dataset. Powerful diagnostic and visualisation tools were developed in order to parameterise the alignment algorithms successfully, and a guide to their use and a showcase of their insight is included.
  
  A low level of noise resides in the volumes generated in Chapter~5, whose frequency and amplitude approximate the dimensions of the microstructure we set out to characterise. Chapter~6 details a novel, robust and highly-performant registration method, analogising Einstein's theory of microscopic diffusion within the realm of differentiable transforms. The algorithm performs remarkably well in damping noise in both a suite of synthetic test volumes and the experimental volumes from Chapter 5, whilst preserving their overall shape. The enhancement facilitates the segmentation of detailed epicardial microstructure from a small section of ventricular wall.
  
  Chapter~7 presents a method for the extraction of anatomical structure from two- and three-dimensional images. Segmentation algorithms are applied to differentiate tissue from background and from interstitial space, and tensor-based image gradients are employed to distinguish interstitial clefts.  The results for the two-dimensional case are promising, highlighting some notable myocardial mesostructure in a section of left-ventricular wall. Progress is made towards the volumetric extraction of interstitial clefts, but are somewhat hindered by tiny aberrant displacements in adjacent slices relative to each other.
  
  Chapter~8 concludes with a summary of the findings of the thesis, and the  trajectories of future work stemming from the roots of progress presented in the previous chapters.
