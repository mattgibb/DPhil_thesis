
General intro ...

\section{Motivation}
\label{sec:intro:motivation}
FROM CONFIRMATION

      This introductory chapter begins to lay out the motivation for the work to be undertaken, from the widest perspective of the fight against heart disease, narrowing down through the computational study of arrhythmia, to the specific role of microstructure in cardiac propagation dynamics and arrhythmogenesis.
      
FROM CONFIRMATION

This thesis is concerned with ...

This is how you cite a reference \cite{citeulike:1448130}

\section{Aims}
\label{sec:intro:aims}
It is the wider aim of the thesis to develop a framework via which the role of microstructure in arrhythmia and electrophysiological pathology may be investigated, based upon information from the latest imaging techniques and leveraging state of the art high performance computing facilities. The chapter concludes by delineating the specific thesis aims and summarising the document structure.

\section{Exegesis}
\label{sec:intro:exegesis}
A summary of the structure of the document.

After this brief introduction, Chapter 2 ...

In Chapter 3 ...

Chapters 4 and 5 investigate ...

...

Conclusions are drawn at the end of each chapter,
but Chapter 8 draws the various threads together to ...
Future work ..


